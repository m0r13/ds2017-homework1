%++++++++++++++++++++++++++++++++++++++++
% Don't modify this section unless you know what you're doing!
\documentclass[letterpaper,12pt]{article}
\usepackage{tabularx} % extra features for tabular environment
\usepackage{amsmath}  % improve math presentation
\usepackage{amsthm}
\usepackage{graphicx} % takes care of graphic including machinery
\usepackage[margin=1in,letterpaper]{geometry} % decreases margins
\usepackage{cite} % takes care of citations
\usepackage[final]{hyperref} % adds hyper links inside the generated pdf file
% more than one optional parameter
\usepackage[usenames]{xcolor}
\usepackage{hyperref}
\usepackage{tikz}
\usepackage{tikz-qtree}
\usepackage[
	n,
	operators,
	advantage,
	sets,
	adversary,
	landau,
	probability,
	notions,	
	logic,
	ff,
	mm,
	primitives,
	events,
	complexity,
	asymptotics,
	keys]{cryptocode}
\usepackage{csquotes}
\usepackage{amsmath,amssymb,booktabs}
\usepackage{fullpage}
\usepackage{dashbox}
\usepackage{todonotes}
\usepackage{url}
\usepackage{float}
\usepackage{flexisym}
\usetikzlibrary{shapes.callouts}
\usetikzlibrary{arrows.meta}
\usetikzlibrary{graphs}
\usetikzlibrary{trees}
\usepackage{listings}
\usepackage{trace}
\usepackage{makeidx}
\usepackage{mdframed}
\usepackage{authblk}
\usepackage{tcolorbox}
\hypersetup{
	colorlinks=true,       % false: boxed links; true: colored links
	linkcolor=blue,        % color of internal links
	citecolor=blue,        % color of links to bibliography
	filecolor=magenta,     % color of file links
	urlcolor=blue
}

% Security parameter in binary and unary representation.
\newcommand{\secp}{n}
\newcommand{\usecp}{1^{\secp}}

\newcommand{\BB}{\mathcal{BB}}

% permutation
\newcommand{\perm}{\pi}
\newcommand{\permj}[1]{\perm_{#1}}

%parties
\newcommand{\Pj}[1]{\mathcal{P}_{#1}}
\newcommand{\Mj}[1]{\mathcal{M}_{#1}}

% Number of repetitions
\newcommand{\nrep}{t}

% Number of senders
\newcommand{\nsender}{N}
\newcommand{\hsender}{H}

% Number of mix-servers.
\newcommand{\nserver}{k}

% Cryptosystem and its algorithms
\newcommand{\CS}{\mathcal{CS}}
\newcommand{\Gen}{\mathsf{Gen}}
\newcommand{\Enc}{\mathsf{Enc}}
\newcommand{\Dec}{\mathsf{Dec}}
\newcommand{\Mspace}[1]{\mathcal{M}_{#1}}
\newcommand{\Rspace}[1]{\mathcal{R}_{#1}}
\newcommand{\Cspace}[1]{\mathcal{C}_{#1}}

% Keys of cryptosystems
%\newcommand{\pk}{\mathit{pk}}
%\newcommand{\sk}{\mathit{sk}}
\newcommand{\pkj}[1]{\pk_{#1}}
\newcommand{\skj}[1]{\sk_{#1}}

\newcommand{\GammaIN}[1]{\Gamma_{#1}^{(\mathit{In})}}
\newcommand{\GammaOUT}[1]{\Gamma_{#1}^{(\mathit{Out})}}

%\newcommand{\sect}[1]{Section~\ref{sect:#1}}
%\newcommand{\app}[1]{Appendix~\ref{sect:#1}}

%\newcommand{\shc}[1]{\textbf{Shahram's comment: #1}}
%\newcommand{\dc}[1]{\textbf{Dog's comment: #1}}

%++++++++++++++++++++++++++++++++++++++++
\title{\vspace{-1.80cm} \small{A short report on:\footnote{This report is prepared in partial fulfillment of the requirements for the course \textit{XYZ} (\href{https://courses.cs.ut.ee}{MTAT.0000}) in Fall 2017, Institute of Computer Science, University of Tartu.}} \\ \vspace{2 mm} \large {\textbf{The title of our report}}}
\author{Author 1, Author 2, Author 3, Author 4}
\affil{\vspace{-0.4cm} University of Tartu, Estonia \\ \href{mailto:xyz@ut.ee}{zyz@ut.ee}}
\date{\vspace{-0.9cm} \scriptsize{\today}}
\renewcommand\Affilfont{\itshape\small}

\begin{document}
\maketitle 
\vspace*{-6 mm}
\begin{abstract}
In this report we ....	

\end{abstract}


\section{Introduction} \label{Intro}


Extractability plays a central role in cryptographic protocol design and analysis. Basically, it relates  to  two-party  protocols  where  one  of  the  parties  (P) has secret input, and tries to convince the other party (V) that it knows the witnesses or secret. The idea is to argue that if V accepts the argument or proof, then P indeed “knows” the secret. More precisely, extractability makes the following requirement: Given access to the internals of any (potentially malicious) P, it is possible to explicitly and efficiently compute (extract) the secret value as long as V accepts an interaction. The notion of \textit{extractable functions} extends the concept of extractability to the more basic setting of computing a function. In this case, convincing a V is replaced by "outputting a value in the range of the function". More specifically, any machine $\mathcal{M}$ that generates a point in the range \textit{knows} a corresponding preimage in the sense that a preimage is efficiently recoverable given the internal state of the machine  \cite{Canetti09-extractable}. 

\begin{figure}[h!]
	\centering
	\includegraphics[width=0.5\linewidth]{Fig1}
	\caption{Extraction by interaction or black-box extraction.}
	\label{fig00}
\end{figure}
%




\subsection{Overview on Questions and Results of the Paper}

The main contribution of the paper is answering to the following questions, 
\begin{enumerate}
	\item \textit{Can one show that extractable functions cannot exist?}
	\item \textit{Can one construct extractable functions from standard hardness assumptions?}
\end{enumerate}


\section{Conclusion}
Due to essential role 




%\section*{References}
\bibliographystyle{alpha}
%\bibliography{bibfile}
%\bibliography{Bibliography}
 \bibliography{Reference1}

\end{document}
